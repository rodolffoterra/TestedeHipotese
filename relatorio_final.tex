\PassOptionsToPackage{unicode=true}{hyperref} % options for packages loaded elsewhere
\PassOptionsToPackage{hyphens}{url}
%
\documentclass[]{article}
\usepackage{lmodern}
\usepackage{amssymb,amsmath}
\usepackage{ifxetex,ifluatex}
\usepackage{fixltx2e} % provides \textsubscript
\ifnum 0\ifxetex 1\fi\ifluatex 1\fi=0 % if pdftex
  \usepackage[T1]{fontenc}
  \usepackage[utf8]{inputenc}
  \usepackage{textcomp} % provides euro and other symbols
\else % if luatex or xelatex
  \usepackage{unicode-math}
  \defaultfontfeatures{Ligatures=TeX,Scale=MatchLowercase}
\fi
% use upquote if available, for straight quotes in verbatim environments
\IfFileExists{upquote.sty}{\usepackage{upquote}}{}
% use microtype if available
\IfFileExists{microtype.sty}{%
\usepackage[]{microtype}
\UseMicrotypeSet[protrusion]{basicmath} % disable protrusion for tt fonts
}{}
\IfFileExists{parskip.sty}{%
\usepackage{parskip}
}{% else
\setlength{\parindent}{0pt}
\setlength{\parskip}{6pt plus 2pt minus 1pt}
}
\usepackage{hyperref}
\hypersetup{
            pdftitle={Existe diferença significante de atrasos de voo entre duas companhia aéreas?},
            pdfauthor={Rodolfo R. Terra \textbar{} Cientista de Dados},
            pdfborder={0 0 0},
            breaklinks=true}
\urlstyle{same}  % don't use monospace font for urls
\usepackage[margin=1in]{geometry}
\usepackage{color}
\usepackage{fancyvrb}
\newcommand{\VerbBar}{|}
\newcommand{\VERB}{\Verb[commandchars=\\\{\}]}
\DefineVerbatimEnvironment{Highlighting}{Verbatim}{commandchars=\\\{\}}
% Add ',fontsize=\small' for more characters per line
\usepackage{framed}
\definecolor{shadecolor}{RGB}{248,248,248}
\newenvironment{Shaded}{\begin{snugshade}}{\end{snugshade}}
\newcommand{\AlertTok}[1]{\textcolor[rgb]{0.94,0.16,0.16}{#1}}
\newcommand{\AnnotationTok}[1]{\textcolor[rgb]{0.56,0.35,0.01}{\textbf{\textit{#1}}}}
\newcommand{\AttributeTok}[1]{\textcolor[rgb]{0.77,0.63,0.00}{#1}}
\newcommand{\BaseNTok}[1]{\textcolor[rgb]{0.00,0.00,0.81}{#1}}
\newcommand{\BuiltInTok}[1]{#1}
\newcommand{\CharTok}[1]{\textcolor[rgb]{0.31,0.60,0.02}{#1}}
\newcommand{\CommentTok}[1]{\textcolor[rgb]{0.56,0.35,0.01}{\textit{#1}}}
\newcommand{\CommentVarTok}[1]{\textcolor[rgb]{0.56,0.35,0.01}{\textbf{\textit{#1}}}}
\newcommand{\ConstantTok}[1]{\textcolor[rgb]{0.00,0.00,0.00}{#1}}
\newcommand{\ControlFlowTok}[1]{\textcolor[rgb]{0.13,0.29,0.53}{\textbf{#1}}}
\newcommand{\DataTypeTok}[1]{\textcolor[rgb]{0.13,0.29,0.53}{#1}}
\newcommand{\DecValTok}[1]{\textcolor[rgb]{0.00,0.00,0.81}{#1}}
\newcommand{\DocumentationTok}[1]{\textcolor[rgb]{0.56,0.35,0.01}{\textbf{\textit{#1}}}}
\newcommand{\ErrorTok}[1]{\textcolor[rgb]{0.64,0.00,0.00}{\textbf{#1}}}
\newcommand{\ExtensionTok}[1]{#1}
\newcommand{\FloatTok}[1]{\textcolor[rgb]{0.00,0.00,0.81}{#1}}
\newcommand{\FunctionTok}[1]{\textcolor[rgb]{0.00,0.00,0.00}{#1}}
\newcommand{\ImportTok}[1]{#1}
\newcommand{\InformationTok}[1]{\textcolor[rgb]{0.56,0.35,0.01}{\textbf{\textit{#1}}}}
\newcommand{\KeywordTok}[1]{\textcolor[rgb]{0.13,0.29,0.53}{\textbf{#1}}}
\newcommand{\NormalTok}[1]{#1}
\newcommand{\OperatorTok}[1]{\textcolor[rgb]{0.81,0.36,0.00}{\textbf{#1}}}
\newcommand{\OtherTok}[1]{\textcolor[rgb]{0.56,0.35,0.01}{#1}}
\newcommand{\PreprocessorTok}[1]{\textcolor[rgb]{0.56,0.35,0.01}{\textit{#1}}}
\newcommand{\RegionMarkerTok}[1]{#1}
\newcommand{\SpecialCharTok}[1]{\textcolor[rgb]{0.00,0.00,0.00}{#1}}
\newcommand{\SpecialStringTok}[1]{\textcolor[rgb]{0.31,0.60,0.02}{#1}}
\newcommand{\StringTok}[1]{\textcolor[rgb]{0.31,0.60,0.02}{#1}}
\newcommand{\VariableTok}[1]{\textcolor[rgb]{0.00,0.00,0.00}{#1}}
\newcommand{\VerbatimStringTok}[1]{\textcolor[rgb]{0.31,0.60,0.02}{#1}}
\newcommand{\WarningTok}[1]{\textcolor[rgb]{0.56,0.35,0.01}{\textbf{\textit{#1}}}}
\usepackage{graphicx,grffile}
\makeatletter
\def\maxwidth{\ifdim\Gin@nat@width>\linewidth\linewidth\else\Gin@nat@width\fi}
\def\maxheight{\ifdim\Gin@nat@height>\textheight\textheight\else\Gin@nat@height\fi}
\makeatother
% Scale images if necessary, so that they will not overflow the page
% margins by default, and it is still possible to overwrite the defaults
% using explicit options in \includegraphics[width, height, ...]{}
\setkeys{Gin}{width=\maxwidth,height=\maxheight,keepaspectratio}
\setlength{\emergencystretch}{3em}  % prevent overfull lines
\providecommand{\tightlist}{%
  \setlength{\itemsep}{0pt}\setlength{\parskip}{0pt}}
\setcounter{secnumdepth}{0}
% Redefines (sub)paragraphs to behave more like sections
\ifx\paragraph\undefined\else
\let\oldparagraph\paragraph
\renewcommand{\paragraph}[1]{\oldparagraph{#1}\mbox{}}
\fi
\ifx\subparagraph\undefined\else
\let\oldsubparagraph\subparagraph
\renewcommand{\subparagraph}[1]{\oldsubparagraph{#1}\mbox{}}
\fi

% set default figure placement to htbp
\makeatletter
\def\fps@figure{htbp}
\makeatother


\title{Existe diferença significante de atrasos de voo entre duas companhia
aéreas?}
\author{Rodolfo R. Terra \textbar{} Cientista de Dados}
\date{}

\begin{document}
\maketitle

Sumário

\ldots{}

1.Coletando os Dados

2.Exploração de Dados

3.Construindo o Dataset

4.Criação de Amostragem

5.Intervalo de Confiança

6.Gráfico dos Intervalos de Confiança

7.Criação do Teste de Hipótese

7.1.Teste t

7.2.Valor p

8.Conclusão

Definição do Problema de Negócio

Para analisar se existe diferença de atraso de vôo entre a companhia
utilizaremos teste de hipósete de um conjunto de dados que possui
336,776 observações e 19 variáveis, chamado flights", que demonstram
dados pontuais de todos os voos que partiram de Nova York em 2013.
Dentre estas companhia escolheremos duas companhias: Delta Airlines (DL)
e a United Airlines (UA).

Etapa 1 - Coletando os Dados

Aqui está a coleta de dados.

\begin{Shaded}
\begin{Highlighting}[]
\CommentTok{# Coletando dados}
\KeywordTok{library}\NormalTok{(}\StringTok{'ggplot2'}\NormalTok{)}
\KeywordTok{library}\NormalTok{(}\StringTok{'dplyr'}\NormalTok{)}
\end{Highlighting}
\end{Shaded}

\begin{verbatim}
## 
## Attaching package: 'dplyr'
\end{verbatim}

\begin{verbatim}
## The following objects are masked from 'package:stats':
## 
##     filter, lag
\end{verbatim}

\begin{verbatim}
## The following objects are masked from 'package:base':
## 
##     intersect, setdiff, setequal, union
\end{verbatim}

\begin{Shaded}
\begin{Highlighting}[]
\KeywordTok{library}\NormalTok{(}\StringTok{'nycflights13'}\NormalTok{)}
\KeywordTok{head}\NormalTok{(flights)}
\end{Highlighting}
\end{Shaded}

\begin{verbatim}
## # A tibble: 6 x 19
##    year month   day dep_time sched_dep_time dep_delay arr_time sched_arr_time
##   <int> <int> <int>    <int>          <int>     <dbl>    <int>          <int>
## 1  2013     1     1      517            515         2      830            819
## 2  2013     1     1      533            529         4      850            830
## 3  2013     1     1      542            540         2      923            850
## 4  2013     1     1      544            545        -1     1004           1022
## 5  2013     1     1      554            600        -6      812            837
## 6  2013     1     1      554            558        -4      740            728
## # ... with 11 more variables: arr_delay <dbl>, carrier <chr>, flight <int>,
## #   tailnum <chr>, origin <chr>, dest <chr>, air_time <dbl>, distance <dbl>,
## #   hour <dbl>, minute <dbl>, time_hour <dttm>
\end{verbatim}

Etapa 2 - Exploração dos Dados

\begin{Shaded}
\begin{Highlighting}[]
\KeywordTok{head}\NormalTok{(flights)}
\end{Highlighting}
\end{Shaded}

\begin{verbatim}
## # A tibble: 6 x 19
##    year month   day dep_time sched_dep_time dep_delay arr_time sched_arr_time
##   <int> <int> <int>    <int>          <int>     <dbl>    <int>          <int>
## 1  2013     1     1      517            515         2      830            819
## 2  2013     1     1      533            529         4      850            830
## 3  2013     1     1      542            540         2      923            850
## 4  2013     1     1      544            545        -1     1004           1022
## 5  2013     1     1      554            600        -6      812            837
## 6  2013     1     1      554            558        -4      740            728
## # ... with 11 more variables: arr_delay <dbl>, carrier <chr>, flight <int>,
## #   tailnum <chr>, origin <chr>, dest <chr>, air_time <dbl>, distance <dbl>,
## #   hour <dbl>, minute <dbl>, time_hour <dttm>
\end{verbatim}

\begin{Shaded}
\begin{Highlighting}[]
\CommentTok{# Visualizando as variáveis}
\KeywordTok{str}\NormalTok{(flights)}
\end{Highlighting}
\end{Shaded}

\begin{verbatim}
## Classes 'tbl_df', 'tbl' and 'data.frame':    336776 obs. of  19 variables:
##  $ year          : int  2013 2013 2013 2013 2013 2013 2013 2013 2013 2013 ...
##  $ month         : int  1 1 1 1 1 1 1 1 1 1 ...
##  $ day           : int  1 1 1 1 1 1 1 1 1 1 ...
##  $ dep_time      : int  517 533 542 544 554 554 555 557 557 558 ...
##  $ sched_dep_time: int  515 529 540 545 600 558 600 600 600 600 ...
##  $ dep_delay     : num  2 4 2 -1 -6 -4 -5 -3 -3 -2 ...
##  $ arr_time      : int  830 850 923 1004 812 740 913 709 838 753 ...
##  $ sched_arr_time: int  819 830 850 1022 837 728 854 723 846 745 ...
##  $ arr_delay     : num  11 20 33 -18 -25 12 19 -14 -8 8 ...
##  $ carrier       : chr  "UA" "UA" "AA" "B6" ...
##  $ flight        : int  1545 1714 1141 725 461 1696 507 5708 79 301 ...
##  $ tailnum       : chr  "N14228" "N24211" "N619AA" "N804JB" ...
##  $ origin        : chr  "EWR" "LGA" "JFK" "JFK" ...
##  $ dest          : chr  "IAH" "IAH" "MIA" "BQN" ...
##  $ air_time      : num  227 227 160 183 116 150 158 53 140 138 ...
##  $ distance      : num  1400 1416 1089 1576 762 ...
##  $ hour          : num  5 5 5 5 6 5 6 6 6 6 ...
##  $ minute        : num  15 29 40 45 0 58 0 0 0 0 ...
##  $ time_hour     : POSIXct, format: "2013-01-01 05:00:00" "2013-01-01 05:00:00" ...
\end{verbatim}

Medidas de Tendência Central da Variável Numéricas

\begin{Shaded}
\begin{Highlighting}[]
\KeywordTok{summary}\NormalTok{(flights}\OperatorTok{$}\NormalTok{arr_delay)}
\end{Highlighting}
\end{Shaded}

\begin{verbatim}
##     Min.  1st Qu.   Median     Mean  3rd Qu.     Max.     NA's 
##  -86.000  -17.000   -5.000    6.895   14.000 1272.000     9430
\end{verbatim}

\begin{Shaded}
\begin{Highlighting}[]
\CommentTok{# Tabela de contigêcia das linhas aéreas}
\KeywordTok{table}\NormalTok{(flights}\OperatorTok{$}\NormalTok{carrier)}
\end{Highlighting}
\end{Shaded}

\begin{verbatim}
## 
##    9E    AA    AS    B6    DL    EV    F9    FL    HA    MQ    OO    UA    US 
## 18460 32729   714 54635 48110 54173   685  3260   342 26397    32 58665 20536 
##    VX    WN    YV 
##  5162 12275   601
\end{verbatim}

\begin{Shaded}
\begin{Highlighting}[]
\CommentTok{#Histograma}
\KeywordTok{hist}\NormalTok{(flights}\OperatorTok{$}\NormalTok{arr_delay, }\DataTypeTok{main =} \StringTok{'Histograma 1'}\NormalTok{, }\DataTypeTok{xlab =} \StringTok{'Atraso de Voo'}\NormalTok{)}
\end{Highlighting}
\end{Shaded}

\includegraphics{relatorio_final_files/figure-latex/summary-1.pdf}

Etapa 3 - Construindo o Dataset

Construção do dataset pop\_data com os dados de voos das companhias
aéreas UA (United Airlines) e DL (Delta Airlines). Contruiremos um
dataset irá conter apenas duas colunas, nome da companhia e atraso nos
voos de chegada. Será conssiderado este dataset como sendo nossa
população de voos:

Metodologia da formula:

\begin{itemize}
\item
  1º eliminação de todos os dados na (Valores missing, não disponivel);
\item
  2º Filtro pela companhia aérea UA (United Airlines) e DL (Delta
  Airlines);
\item
  3º Filtro para retornar apenas os valores que forem igual ou maior a
  `Zero'. Os valores negativos serão descontiderádos;
\end{itemize}

\begin{Shaded}
\begin{Highlighting}[]
\NormalTok{pop_data =}\StringTok{ }\KeywordTok{na.omit}\NormalTok{(flights) }\OperatorTok
\StringTok{  }\KeywordTok{filter}\NormalTok{(carrier }\OperatorTok{==}\StringTok{ 'UA'} \OperatorTok{|}\StringTok{ }\NormalTok{carrier }\OperatorTok{==}\StringTok{ 'DL'}\NormalTok{, arr_delay }\OperatorTok{>=}\DecValTok{0}\NormalTok{) }\OperatorTok
\StringTok{  }\KeywordTok{select}\NormalTok{(carrier, arr_delay) }\OperatorTok
\StringTok{  }\KeywordTok{group_by}\NormalTok{(carrier) }\OperatorTok
\StringTok{  }\KeywordTok{sample_n}\NormalTok{(}\DecValTok{17000}\NormalTok{) }\OperatorTok
\StringTok{  }\KeywordTok{ungroup}\NormalTok{()}


\KeywordTok{head}\NormalTok{(pop_data)}
\end{Highlighting}
\end{Shaded}

\begin{verbatim}
## # A tibble: 6 x 2
##   carrier arr_delay
##   <chr>       <dbl>
## 1 DL             17
## 2 DL              4
## 3 DL              5
## 4 DL              7
## 5 DL            235
## 6 DL             87
\end{verbatim}

\begin{Shaded}
\begin{Highlighting}[]
\KeywordTok{tail}\NormalTok{(pop_data)}
\end{Highlighting}
\end{Shaded}

\begin{verbatim}
## # A tibble: 6 x 2
##   carrier arr_delay
##   <chr>       <dbl>
## 1 UA              7
## 2 UA             13
## 3 UA             14
## 4 UA             39
## 5 UA              5
## 6 UA             10
\end{verbatim}

\begin{Shaded}
\begin{Highlighting}[]
\KeywordTok{hist}\NormalTok{(pop_data}\OperatorTok{$}\NormalTok{arr_delay, }\DataTypeTok{main =} \StringTok{'Histograma 2'}\NormalTok{, }\DataTypeTok{xlab =} \StringTok{'Atraso de Voo'}\NormalTok{, }\DataTypeTok{ylab=}\StringTok{"Frequencia"}\NormalTok{)}
\end{Highlighting}
\end{Shaded}

\includegraphics{relatorio_final_files/figure-latex/modelagem-1.pdf}

É importante observar que neste histograma removemos os valores
negativos, pois temos como principal objetivo observar somente os
atrasos de voos. Os valores no conjunto de dados que representam menores
que zero seguinifica que o voo chegou a sei destido antes da data
prevista, então não houve atraso.

Etapa 4 - Criação de Amostragem

Criação de duas amostras de 1000 observações cada uma a partir do
dataset pop\_data apenas com dados da companhia DL para amostra 1 e
apenas dados da companhia UA na amostra 2.

\begin{Shaded}
\begin{Highlighting}[]
\NormalTok{amostra1 <-}\StringTok{ }\KeywordTok{na.omit}\NormalTok{(pop_data) }\OperatorTok\StringTok{ }
\StringTok{  }\KeywordTok{select}\NormalTok{(carrier, arr_delay) }\OperatorTok
\StringTok{  }\KeywordTok{filter}\NormalTok{(carrier }\OperatorTok{==}\StringTok{ 'DL'}\NormalTok{) }\OperatorTok
\StringTok{  }\KeywordTok{mutate}\NormalTok{(}\DataTypeTok{sample_id =} \StringTok{'1'}\NormalTok{) }\OperatorTok
\StringTok{  }\KeywordTok{sample_n}\NormalTok{(}\DecValTok{1000}\NormalTok{)}
                      
\KeywordTok{head}\NormalTok{(amostra1)}
\end{Highlighting}
\end{Shaded}

\begin{verbatim}
## # A tibble: 6 x 3
##   carrier arr_delay sample_id
##   <chr>       <dbl> <chr>    
## 1 DL              7 1        
## 2 DL             18 1        
## 3 DL             16 1        
## 4 DL              1 1        
## 5 DL             25 1        
## 6 DL              6 1
\end{verbatim}

\begin{Shaded}
\begin{Highlighting}[]
\NormalTok{amostra2 <-}\StringTok{ }\KeywordTok{na.omit}\NormalTok{(pop_data) }\OperatorTok
\StringTok{  }\KeywordTok{select}\NormalTok{(carrier, arr_delay) }\OperatorTok
\StringTok{  }\KeywordTok{filter}\NormalTok{(carrier }\OperatorTok{==}\StringTok{ "UA"}\NormalTok{) }\OperatorTok
\StringTok{  }\KeywordTok{mutate}\NormalTok{(}\DataTypeTok{sample_id =} \StringTok{"2"}\NormalTok{) }\OperatorTok
\StringTok{  }\KeywordTok{sample_n}\NormalTok{(}\DecValTok{1000}\NormalTok{)}

\KeywordTok{head}\NormalTok{(amostra2)}
\end{Highlighting}
\end{Shaded}

\begin{verbatim}
## # A tibble: 6 x 3
##   carrier arr_delay sample_id
##   <chr>       <dbl> <chr>    
## 1 UA             13 2        
## 2 UA             12 2        
## 3 UA             15 2        
## 4 UA              3 2        
## 5 UA              8 2        
## 6 UA              7 2
\end{verbatim}

\begin{Shaded}
\begin{Highlighting}[]
\CommentTok{# Criação de um dataset contendo os dados das 2 amostras criadas no item anterior. }

\NormalTok{samples =}\StringTok{ }\KeywordTok{rbind}\NormalTok{(amostra1, amostra2)}
\KeywordTok{head}\NormalTok{(samples)}
\end{Highlighting}
\end{Shaded}

\begin{verbatim}
## # A tibble: 6 x 3
##   carrier arr_delay sample_id
##   <chr>       <dbl> <chr>    
## 1 DL              7 1        
## 2 DL             18 1        
## 3 DL             16 1        
## 4 DL              1 1        
## 5 DL             25 1        
## 6 DL              6 1
\end{verbatim}

\begin{Shaded}
\begin{Highlighting}[]
\KeywordTok{tail}\NormalTok{(samples)}
\end{Highlighting}
\end{Shaded}

\begin{verbatim}
## # A tibble: 6 x 3
##   carrier arr_delay sample_id
##   <chr>       <dbl> <chr>    
## 1 UA              8 2        
## 2 UA              4 2        
## 3 UA             72 2        
## 4 UA            115 2        
## 5 UA              5 2        
## 6 UA              6 2
\end{verbatim}

Etapa 5 - Intervalo de Confiança

Calculo do intervalo de confiança (95\%) da amostra1.

Fórmula de erro padrão: erro\_padrao = sd(amostra\$arr\_delay) /
sqrt(nrow(amostra))

Esta fórmula é usada para calcular o desvio padrão de uma distribuição
da média amostral (de um grande número de amostras de uma população). Em
outras palavras, só é aplicável quando você está procurando o desvio
padrão de médias calculadas a partir de uma amostra de tamanho n𝑛,
tirada de uma população.

Digamos que você obtenha 10000 amostras de uma população qualquer com um
tamanho de amostra de n = 2. Então calculamos as médias de cada uma
dessas amostras (teremos 10000 médias calculadas). A equação acima
informa que, com um número de amostras grande o suficiente, o desvio
padrão das médias da amostra pode ser aproximado usando esta fórmula:
sd(amostra) / sqrt(nrow(amostra))

Deve ser intuitivo que o seu desvio padrão das médias da amostra será
muito pequeno, ou em outras palavras, as médias de cada amostra terão
muito pouca variação.

Com determinadas condições de inferência (nossa amostra é aleatória,
normal, independente), podemos realmente usar esse cálculo de desvio
padrão para estimar o desvio padrão de nossa população. Como isso é
apenas uma estimativa, é chamado de erro padrão. A condição para usar
isso como uma estimativa é que o tamanho da amostra n é maior que 30
(dado pelo teorema do limite central) e atende a condição de
independência n \textless{}= 10\% do tamanho da população.

amostra 1

Erro Padrão

\begin{Shaded}
\begin{Highlighting}[]
\NormalTok{erro_padrao_amostra1 =}\StringTok{ }\KeywordTok{sd}\NormalTok{(amostra1}\OperatorTok{$}\NormalTok{arr_delay) }\OperatorTok{/}\StringTok{ }\KeywordTok{sqrt}\NormalTok{(}\KeywordTok{nrow}\NormalTok{(amostra1))}
\end{Highlighting}
\end{Shaded}

Limites Inferior e Superior

1.96 é o valor de z score para 95\% de confiança

\begin{Shaded}
\begin{Highlighting}[]
\NormalTok{lower1 =}\StringTok{ }\KeywordTok{mean}\NormalTok{(amostra1}\OperatorTok{$}\NormalTok{arr_delay) }\OperatorTok{-}\StringTok{ }\FloatTok{1.96} \OperatorTok{*}\StringTok{ }\NormalTok{erro_padrao_amostra1}
\NormalTok{upper1 =}\StringTok{ }\KeywordTok{mean}\NormalTok{(amostra1}\OperatorTok{$}\NormalTok{arr_delay) }\OperatorTok{+}\StringTok{ }\FloatTok{1.96} \OperatorTok{*}\StringTok{ }\NormalTok{erro_padrao_amostra1}
\end{Highlighting}
\end{Shaded}

Intervalo de Confiança

\begin{Shaded}
\begin{Highlighting}[]
\NormalTok{ic_}\DecValTok{1}\NormalTok{ =}\StringTok{ }\KeywordTok{c}\NormalTok{(lower1, upper1)}
\KeywordTok{mean}\NormalTok{(amostra1}\OperatorTok{$}\NormalTok{arr_delay)}
\end{Highlighting}
\end{Shaded}

\begin{verbatim}
## [1] 34.849
\end{verbatim}

\begin{Shaded}
\begin{Highlighting}[]
\NormalTok{ic_}\DecValTok{1}
\end{Highlighting}
\end{Shaded}

\begin{verbatim}
## [1] 31.41951 38.27849
\end{verbatim}

amostra 2

Erro Padrão

\begin{Shaded}
\begin{Highlighting}[]
\NormalTok{erro_padrao_amostra2 =}\StringTok{ }\KeywordTok{sd}\NormalTok{(amostra1}\OperatorTok{$}\NormalTok{arr_delay) }\OperatorTok{/}\StringTok{ }\KeywordTok{sqrt}\NormalTok{(}\KeywordTok{nrow}\NormalTok{(amostra2))}
\end{Highlighting}
\end{Shaded}

Limites Inferior e Superior

1.96 é o valor de z score para 95\% de confiança

\begin{Shaded}
\begin{Highlighting}[]
\NormalTok{lower2 =}\StringTok{ }\KeywordTok{mean}\NormalTok{(amostra2}\OperatorTok{$}\NormalTok{arr_delay) }\OperatorTok{-}\StringTok{ }\FloatTok{1.96} \OperatorTok{*}\StringTok{ }\NormalTok{erro_padrao_amostra2}
\NormalTok{upper2 =}\StringTok{ }\KeywordTok{mean}\NormalTok{(amostra2}\OperatorTok{$}\NormalTok{arr_delay) }\OperatorTok{+}\StringTok{ }\FloatTok{1.96} \OperatorTok{*}\StringTok{ }\NormalTok{erro_padrao_amostra2}
\end{Highlighting}
\end{Shaded}

Intervalo de Confiança

\begin{Shaded}
\begin{Highlighting}[]
\NormalTok{ic_}\DecValTok{2}\NormalTok{ =}\StringTok{ }\KeywordTok{c}\NormalTok{(lower2, upper2)}
\KeywordTok{mean}\NormalTok{(amostra1}\OperatorTok{$}\NormalTok{arr_delay)}
\end{Highlighting}
\end{Shaded}

\begin{verbatim}
## [1] 34.849
\end{verbatim}

\begin{Shaded}
\begin{Highlighting}[]
\NormalTok{ic_}\DecValTok{2}
\end{Highlighting}
\end{Shaded}

\begin{verbatim}
## [1] 29.99951 36.85849
\end{verbatim}

Etapa 6 - Gráfico dos Intervalos de Confianças

Criação de um plot Visualizando os intervalos de confiança criados nos
itens anteriores.

\begin{Shaded}
\begin{Highlighting}[]
\NormalTok{toPlot =}\StringTok{ }\KeywordTok{summarise}\NormalTok{(}\KeywordTok{group_by}\NormalTok{(samples, sample_id), }\DataTypeTok{mean =} \KeywordTok{mean}\NormalTok{(arr_delay))}
\NormalTok{toPlot =}\StringTok{ }\KeywordTok{mutate}\NormalTok{(toPlot, }\DataTypeTok{lower =} \KeywordTok{ifelse}\NormalTok{(toPlot}\OperatorTok{$}\NormalTok{sample_id }\OperatorTok{==}\StringTok{ }\DecValTok{1}\NormalTok{,ic_}\DecValTok{1}\NormalTok{[}\DecValTok{1}\NormalTok{],ic_}\DecValTok{2}\NormalTok{[}\DecValTok{1}\NormalTok{]))}
\NormalTok{toPlot =}\StringTok{ }\KeywordTok{mutate}\NormalTok{(toPlot, }\DataTypeTok{upper =} \KeywordTok{ifelse}\NormalTok{(toPlot}\OperatorTok{$}\NormalTok{sample_id }\OperatorTok{==}\StringTok{ }\DecValTok{1}\NormalTok{,ic_}\DecValTok{1}\NormalTok{[}\DecValTok{2}\NormalTok{],ic_}\DecValTok{2}\NormalTok{[}\DecValTok{2}\NormalTok{]))}
\KeywordTok{ggplot}\NormalTok{(toPlot, }\KeywordTok{aes}\NormalTok{(}\DataTypeTok{x =}\NormalTok{ sample_id, }\DataTypeTok{y=}\NormalTok{mean, }\DataTypeTok{colour =}\NormalTok{ sample_id)) }\OperatorTok{+}\StringTok{ }
\StringTok{  }\KeywordTok{geom_point}\NormalTok{() }\OperatorTok{+}
\StringTok{  }\KeywordTok{geom_errorbar}\NormalTok{(}\KeywordTok{aes}\NormalTok{(}\DataTypeTok{ymin=}\NormalTok{lower, }\DataTypeTok{ymax=}\NormalTok{upper), }\DataTypeTok{width=}\NormalTok{.}\DecValTok{1}\NormalTok{)}
\end{Highlighting}
\end{Shaded}

\includegraphics{relatorio_final_files/figure-latex/grafico-1.pdf}

A maior parte dos dados reside no mesmo intervalo de confiança nas duas
amostras. Motimo pelo qual podemos dizer que muito provavelmente, as
amostras vieram da mesma população.

Etapa 7 - Criação do Teste de Hipótese

Crição de um teste de hipótese para verificar se os voos da Delta
Airlines (DL) atrasam mais do que os voos da UA (United Airlines).

H0 e H1 devem ser mutuamente exclusivas.

\begin{itemize}
\tightlist
\item
  H0 = Não há diferença significativa entre os atrasos da DL e UA (diff
  da média de atrasos = 0).
\item
  H1 = Delta atrasa mais (diff das médias \textgreater{} 0).
\end{itemize}

 Criação das Amostras

\begin{Shaded}
\begin{Highlighting}[]
\NormalTok{dl <-}\StringTok{ }\KeywordTok{sample_n}\NormalTok{(}\KeywordTok{filter}\NormalTok{(pop_data, carrier }\OperatorTok{==}\StringTok{ "DL"}\NormalTok{, arr_delay }\OperatorTok{>}\StringTok{ }\DecValTok{0}\NormalTok{), }\DecValTok{1000}\NormalTok{)}
\NormalTok{ua <-}\StringTok{ }\KeywordTok{sample_n}\NormalTok{(}\KeywordTok{filter}\NormalTok{(pop_data, carrier }\OperatorTok{==}\StringTok{ "UA"}\NormalTok{, arr_delay }\OperatorTok{>}\StringTok{ }\DecValTok{0}\NormalTok{), }\DecValTok{1000}\NormalTok{)}
\end{Highlighting}
\end{Shaded}

Calculo do Erro Padrão Médio

Amostra 1

\begin{Shaded}
\begin{Highlighting}[]
\NormalTok{se1 =}\StringTok{ }\KeywordTok{sd}\NormalTok{(dl}\OperatorTok{$}\NormalTok{arr_delay) }\OperatorTok{/}\StringTok{ }\KeywordTok{sqrt}\NormalTok{(}\KeywordTok{nrow}\NormalTok{(dl))}
\KeywordTok{mean}\NormalTok{(dl}\OperatorTok{$}\NormalTok{arr_delay)}
\end{Highlighting}
\end{Shaded}

\begin{verbatim}
## [1] 35.624
\end{verbatim}

Limites Inferior e Superior

\begin{Shaded}
\begin{Highlighting}[]
\NormalTok{lower11 =}\StringTok{ }\KeywordTok{mean}\NormalTok{(dl}\OperatorTok{$}\NormalTok{arr_delay) }\OperatorTok{-}\StringTok{ }\FloatTok{1.96} \OperatorTok{*}\StringTok{ }\NormalTok{se1}
\NormalTok{upper11 =}\StringTok{ }\KeywordTok{mean}\NormalTok{(dl}\OperatorTok{$}\NormalTok{arr_delay) }\OperatorTok{+}\StringTok{ }\FloatTok{1.96} \OperatorTok{*}\StringTok{ }\NormalTok{se1}
\NormalTok{ic_dl =}\StringTok{ }\KeywordTok{c}\NormalTok{(lower11,upper11)}
\NormalTok{ic_dl}
\end{Highlighting}
\end{Shaded}

\begin{verbatim}
## [1] 31.85288 39.39512
\end{verbatim}

Calcula erro padrão e média Amostra 2

\begin{Shaded}
\begin{Highlighting}[]
\NormalTok{se2 =}\StringTok{ }\KeywordTok{sd}\NormalTok{(ua}\OperatorTok{$}\NormalTok{arr_delay) }\OperatorTok{/}\StringTok{ }\KeywordTok{sqrt}\NormalTok{(}\KeywordTok{nrow}\NormalTok{(ua))}
\KeywordTok{mean}\NormalTok{(ua}\OperatorTok{$}\NormalTok{arr_delay)}
\end{Highlighting}
\end{Shaded}

\begin{verbatim}
## [1] 36.246
\end{verbatim}

Limites inferior e superior

\begin{Shaded}
\begin{Highlighting}[]
\NormalTok{lower22 =}\StringTok{ }\KeywordTok{mean}\NormalTok{(ua}\OperatorTok{$}\NormalTok{arr_delay) }\OperatorTok{-}\StringTok{ }\FloatTok{1.96} \OperatorTok{*}\StringTok{ }\NormalTok{se2}
\NormalTok{upper22 =}\StringTok{ }\KeywordTok{mean}\NormalTok{(ua}\OperatorTok{$}\NormalTok{arr_delay) }\OperatorTok{+}\StringTok{ }\FloatTok{1.96} \OperatorTok{*}\StringTok{ }\NormalTok{se2}
\NormalTok{ic_ua =}\StringTok{ }\KeywordTok{c}\NormalTok{(lower22,upper22)}
\NormalTok{ic_ua}
\end{Highlighting}
\end{Shaded}

\begin{verbatim}
## [1] 33.09175 39.40025
\end{verbatim}

Etapa 7.1 - Teste T

O teste t (de Student) foi desenvolvido por Willian Sealy Gosset em 1908
que usou o pseudônimo ``Student'' em função da confidencialidade
requerida por seu empregador (cervejaria Guiness) que considerava o uso
de estatística na manutenção da qualidade como uma vantagem competitiva.
O teste t de Student tem diversas variações de aplicação, e pode ser
usado na comparação de duas (e somente duas) médias e as variações dizem
respeito às hipóteses que são testadas.

\begin{Shaded}
\begin{Highlighting}[]
\KeywordTok{t.test}\NormalTok{(dl}\OperatorTok{$}\NormalTok{arr_delay, ua}\OperatorTok{$}\NormalTok{arr_delay, }\DataTypeTok{alternative=}\StringTok{"greater"}\NormalTok{)}
\end{Highlighting}
\end{Shaded}

\begin{verbatim}
## 
##  Welch Two Sample t-test
## 
## data:  dl$arr_delay and ua$arr_delay
## t = -0.24797, df = 1937.5, p-value = 0.5979
## alternative hypothesis: true difference in means is greater than 0
## 95 percent confidence interval:
##  -4.749843       Inf
## sample estimates:
## mean of x mean of y 
##    35.624    36.246
\end{verbatim}

Etapa 7.2 - Valor p

O valor-p é uma quantificação da probabilidade de se errar ao rejeitar
H0 e a mesma decorre da distribuição estatística adotada. Se o valor-p é
menor que o nível de significância, conclui-se que o correto é rejeitar
a hipótese de nulidade.

Valor p é a probabiblidade de que a estatística do teste assuma um valor
extremo em relação ao valor observado quando H0 é verdadeira.

Estamos trabalhando com alfa igual a 0.05 (95\% de confiança)

Regra

\begin{itemize}
\tightlist
\item
  Baixo valor p: forte evidência empírica contra h0
\item
  Alto valor p: pouca ou nenhuma evidência empírica contra h0

   
\end{itemize}

Etapa 8 - Conclusão

\begin{itemize}
\tightlist
\item
  Falhamos em rejeitar a hipótese nula, pois p-valor é maior que o nível
  de significância.
\item
  Isso que dizer que há uma probabilidade alta de não haver diferença
  significativa entre os atrasos.
\item
  Para os nossos dados, não há evidência estatística de que a DL atrase
  mais que a UA.
\end{itemize}

\ldots{}

Dados Pessoais

Site www.rodolfoterra.com

Linkedin rodolffoterra

Repertório no GitHub: Teste de Hipotese

E-mail
\href{mailto:consultoriaterra@hotmail.com}{\nolinkurl{consultoriaterra@hotmail.com}}

\end{document}
